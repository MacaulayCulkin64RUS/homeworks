\documentclass[bachelor,och,referat]{SCWorks_corrected}
\usepackage{graphicx} 
\usepackage[english,russian]{babel}
\usepackage[sort,compress]{cite}
\usepackage{amsmath}
\usepackage{amssymb}
\usepackage{amsthm}
\usepackage{fancyvrb}
\usepackage{longtable}
\usepackage{array}
\usepackage[english,russian]{babel}
\usepackage{tempora}
\usepackage{url}
\usepackage{hyperref}


\newcommand{\eqdef}{\stackrel {\rm def}{=}}

\newtheorem{lem}{Лемма}



\begin{document}
\chair{информатики и программирования}
\title{Искусственный интеллект}
\course{1}
\group{111}
\napravlenie{02.03.02 "--- Фундаментальная информатика и информационные технологии}
\author{Иллипурова Константина Дмитриевича}
\chtitle{к.\,ф.-м.\,н.}
\chname{С.\,В.\,Миронов}
\satitle{доцент, к.\,ф.-м.\,н.}
\saname{А.\,П.\,Грецова}
\date{2023}

\maketitle
\tableofcontents

\intro

Искусственный интеллект --- это область информатики и компьютерной науки, которая занимается созданием программ и систем, способных обучаться и выполнять различные задачи вместо человека.\cite{N1}

Искусственный интеллект связан со сходной задачей использования компьютеров для понимания человеческого интеллекта, но не обязательно ограничивается биологически правдоподобными методами.

И хотя ИИ имеет очень узкие области применения: решить математический пример или обыграть гроссмейстера в шахматы, уже сегодня мы видим что их функционал расширяется.

Тренд, появившийся с выпуском Chat GPT совершил то, что ни одна другая сеть не могла: стать общеизвестной. Теперь, люди даже далекие от IT-сферы заинтересовались этой областью, а значит что на них, то есть нейросети есть спрос. В ближайшие несколько лет разработка нейросетей станет если не лидирующим направлением в обучении, то одним из самых популярных. А о инвестициях, которые теперь вкладываются в эту сферу нечего и говорить.

Художники начали пользоваться такими ИИ, как, к примеру Midjourney; те, кто отказывался от <<привычных>> заметок, начинали пользоваться Notion'ом, а студент РГГУ защитил дипломную работу с помощью нейросети.\cite{N7}

Искусственный интеллект имеет потенциал для решения множества проблем и улучшения качества жизни во многих областях, и поэтому является одним из самых важных направлений развития современной технологии.

\section{История развития ИИ}

Точкой отсчета в истории развития ИИ можно обозначить начало 16 века, когда немецкий ученый Вильгельм Шиккард создал первое в мире счетное устройство, автоматически выполняющее сложение, вычитание, умножение и деление. Тогда же математик Рене Декарт представил гипотезу, что животные – это механизмы, но мыслящие, а потому у нашей цивилизации есть важная задача: изобрести умную машину, обладающую интеллектом.\cite{N11}

Огромный скачок в развитии искусственного интеллекта произошел в XX веке. Торрес Кеведо в 1914 году изобрел устройство, способное играть в шахматы. Конрад Цузе в 1941 году построил первый программно-управляемый компьютер. Ученые в области физиологии и психологии разработали теории о работе человеческого мозга. Уоррен Мак-Каллок и Уолтер Питтс в статье <<Логические исчисления идей, относящихся к нервной активности>> предложили модель искусственного нейрона. Дональд Хебб в работе <<Организация поведения>> описал принципы обучения нейронов. Фрэнк Розенблатт предложил математическую модель восприятия информации мозгом.

Алан Тьюринг в 1950 году написал статью <<Может ли машина мыслить?>>, в которой он рассказал о процедуре, способной определить момент, когда машина сравняется в интеллектуальном плане с человеком. Данную процедуру назвали тестом Тьюринга.

В 1956 году Джон Маккарти, Марвин Мински, Клод Шеннон и Натаниэль Рочестер организовали семинар в Дартмуте. Здесь произошло формирование Искусственного интеллекта как науки. В ходе семинара был сформулирован один из основных принципов создания искусственного интеллекта: нужно создать машину, способную продемонстрировать один из видов обучения. Этот принцип был предложен Марвином Минским. Данная машина должна быть снабжена средствами обеспечения разнообразных выходных ответов на входящие вопросы. 

Также ученые поняли, что машина должна обладать языком, так как он является уровнем высшей абстракции. Ученые видели необходимость в обучении машины языку программирования.

А. Ньюэлл, Дж. Шоу и Г. Саймон создали программу для игры в шахматы. А. Самюэл написал программы для игры в шашки. В 1960 году группа голландских психологов под руководством А. де Гроота разработала программу, способную решать некоторые головоломки,находить неопределенный интеграл. В 1963 году метод автоматического доказательства теорем был реализован Д. Робинсоном. В 1965 году Э. Фейгенбаум разработал первую экспертную систему. В 70-е годы происходит разработка компьютерных программ, использующих базы данных для решения задач.

Сейчас происходит качественный переход от вычислительной эры к эре когнитивной (в терминах футурологов, Second Machine Age), когда компьютеры нового типа быстро учатся работать со структурированными, неструктурированными и нечетко структурированными данными, начинают замещать труд людей при решении большого количества когнитивных задач.\cite{N9}

Массовое распространение смартфонов породило широкое использование речевых помощников, в которых реализуются элементы ИИ. Такие приложения помогают пользователю в его повседневной деятельности. Среди них такие известные приложения, как Siri (компании Apple), Cortana (Microsoft), Google Now (Google), Echo (Amazon), «Алиса» (Яндекс) и др., которым уже пользуются десятки миллионов людей. Данные приложения реализуются также на планшетах, ноутбуках и персональных компьютерах. Со временем эти программы станут все интеллектуальнее и незаменимее.

Важным направлением работ по ИИ является выявление структуры мозга человека. Такие проекты весьма дорогостоящие, и потому их реализацию могут позволить себе немногие страны и гигантские корпорации. 

Развитием технологии глубинного обучения стала реализованная IBM летом 2017 г. технология распределенного глубинного обучения (DDL), позволяющая на порядок сократить время обучения искусственной нейронной сети.

Будущие ИИ-тренды обещают больше возможностей, чем любые другие технологические тренды. Этому существуют убедительные подтверждения; невозможно игнорировать наличие интеллекта, демонстрируемого машинами.

Наибольшее число молодых инновативных фирм, разрабатывающих ИИ, находятся в США, Европе, Китае, Израиле, Великобритании, Канаде. Среди компаний, зарегистрировавших наибольшее число патентов в области ИИ находятся IBM, Microsoft, Toshiba, Samsung, NEC, Fujitsu, Hitachi, Panasonic, Canon

В нашей стране направление «Искусственный интеллект» возникло в 60-х годах XX века. Ученые рассматривали искусственный интеллект как возможность решать прикладные задачи. В 1988 году была создана Российская ассоциация искусственного интеллекта.\cite{N12}

Российские разработчики Владимир Веселов, Евгений Демченко и Сергей Уласень разработали программу чат-бота Eugene Goostman , которая в 2014 году смогла пройти тест Тьюинга, обманув жюри, считавших, что они общаются с человеком, на 33 процента.

В настоящее время разработки в сфере искусственного интеллекта идут в следующих направлениях:\cite{N2} 
\begin{itemize} 
    \item компьютерное зрение
    \item обработка естественного языка
    \item распознавание и синтез речи
    \item интеллектуальные системы поддержки принятия решений
    \item перспективные методы ИИ.
\end{itemize} 
Большую роль в области развития искусственного интеллекта сыграла военная отрасль, спонсирующая разработку новых систем.

Сейчас разработано множество систем искусственного интеллекта, внедренных в различные сферы человеческой деятельности. Однако искусственный интеллект, способный заменить человека во всех областях жизни, еще не был создан.

\section{Классификация искусственного интеллекта}

Необходимость классификации вызвана потому, что ни одна существующая система не может называться ИИ в классическом смысле. Рассмотрим отдельные понимания того, что называется ИИ.\cite{N3}

Расположим отдельные термины ИИ сверху вниз, где предположим, что чем <<выше>> расположение, тем <<выше>> показатель <<интеллекта>>.\begin{itemize}
    \item Strong AI
    \item AGI
    \item Wide AI
    \item Narrow AI
    \item Basic algorithms    
\end{itemize}

К самым простым интеллектуальным системам можно отнести системы с обратной связью. То есть системы, которые принимают некоторые сигналы от внешнего мира и подстраивают свою работу под изменяющиеся условия.

На май 2023 года можно уверенно сказать, что все существующие системы ИИ относятся к группе Narrow AI (узкий или слабый ИИ), потому, что задачи каждой из нейросетей скатывается к решении одной определенной задачи, в строго заданных граничных условиях. Хотя, во многих случаях, нейросети справляются лучше и быстрее, чем человек, все равно узконаправленность играет роль.

Можно выделить следующие проблемы развития ИИ на сегодняшний день:\begin{itemize}
    \item Существует вариант развития такой, что если ИИ не остановится на человеческом уровне, а будет развиваться дальше, то развитие будет неконтролируемым, а значит мы упираемся в отсутствие теорий и моделей интеллекта.
    \item Если интеллект строит модели реальности, то построение всё более сложных моделей потребует всё больших (небезграничных) ресурсов
    \item отсутствует запоминание ранее приобретенных навыков при обучении новым; 
    \item ИИ не может при обучении новым навыкам опираться на ранее приобретенные, т.е. отсутствует обобщение накопленных знаний и использование их в разных контекстах.\cite{N9}
\end{itemize}

Следующая ступень, к которой сейчас продвигаются многие исследователи искусственного интеллекта, "--- Wide AI. Главная задача таких систем — преодолеть узость применимости. Есть список проблем, которые должны быть решены в таких системах. Вот только некоторые из них:\begin{itemize}
    \item объяснимость ИИ (explainable AI) — должно быть понимание принимаемых ИИ решений. Только понимая механизмы вывода, можно верифицировать, дорабатывать и дообучать ИИ;
    \item передача знаний (transfer learning) — две системы должны иметь возможность обучать и дообучать друг друга, определять разность знаний и противоречия в них;
    \item быстрое обучение (few-shot learning) — система должна обучаться на небольшом объеме материала и за малое число итераций;
    \item решение проблемы катастрофического забывания (structured prediction and learning) — при дообучении система не должна терять уже имеющиеся способности.
    \item инкрементальное обучение (incremental learning) — система должна иметь возможность постоянно накапливать знания, интегрируя их с уже изученными.
\end{itemize}

Как видно, цель этого шага — расширить возможности существующих систем ИИ, позволить им постоянно обучаться, обмениваться знаниями и расширять границы применения в решении все еще достаточно <<узких>> задач.\cite{N3}

Следующий этап развития "--- общий искусственный интеллект (AGI). AGI – это способность системы решать любые задачи в сложных средах с ограниченными ресурсами.

Это определение максимально широкое и требует дополнительного уточнения, что такое любые задачи, какими должны быть сложные среды, о каких ресурсах идёт речь и насколько они ограничены. Увы, нет четкого понимания, что представляет собой такая система. Упрощенно можно сказать, что по своим интеллектуальным способностям она должна находиться на уровне человека. И здесь возникает вопрос уже не как этого добиться, а как это проверить. 

Во"=первых, нужно выработать некоторый набор критериев, достижение которых будет означать, что мы создали AGI. Во"=вторых, было бы хорошо иметь некоторые числовые показатели, по которым можно сравнить две системы искусственного интеллекта и определить, какая из них более интеллектуальна. Обе задачи до сих пор не решены даже приблизительно.

Действительно хорошего практического решения проблемы метрики развития AGI в настоящее время не существует. Некоторые авторы полагают, что измерение частичного прогресса в AGI в целом крайне проблематично, упрощенно говоря, потому что система AGI не будет проявлять свойств AGI до тех пор, пока она не будет построена целиком. Но попытки разработать такую метрику принимаются. 

Еще больше вопросов возникает с последним этапом развития ИИ "--- сильным искусственным интеллектом. Часто этот термин употребляют в значении AGI. Но правильнее будет охарактеризовать такой интеллект как многократно превосходящий по уровню развития человеческий. 

Мы находимся лишь в начале пути к сильному искусственному интеллекту, и еще предстоит решить множество теоретических и практических проблем его понимания и создания.

\section{Применение ИИ в различных отраслях}

\subsection{Применение искусственного интеллекта в медицине}
    
Для этой сферы применения технологий искусственного интеллекта особенно актуальны его способности собирать, анализировать информацию и делать логичные заключения. Благодаря этому ИИ можно использовать для постановки диагноза, регистрации данных, выполнения функции ассистента врача. Кроме этих обязанностей, ИИ можно поручить определение предрасположенности пациента к развитию конкретных патологий, прогнозирование течения заболеваний хронического типа, раннее выявление болезни.\cite{N4}

Такие программы уже запущены на суперкомпьютере Watson от IBM, DeepMind Healthot Google, разрабатывается приложение Face2Gene от FDNA (определение болезней, передающихся генетическим путем, по фотографии). В РФ продолжается работа над системой поддержки принятия решений с использованием ИИ --- Третье мнение. В онкологических центрах пользуются программой Botkin.Al.

Особая роль у систем, занятых созданием новых препаратов. Сегодня на разработку вместе с запуском в продажу современного лекарственного средства, как утверждает топ-менеджер Pfizer Джуди Сюардс, уходит примерно двенадцать лет.

Благодаря ИИ время, затрачиваемое на построение молекулярной структуры и моделирование препарата, значительно сокращается с одновременным повышением качества. Первые суперкомпьютеры, способные решить эту задачу, начали создавать специалисты компаний Atomwise и Berg Health.\cite{N13}

\subsection{Применение искусственного интеллекта в сфере образования}
    
На сегодняшний день перед сферой образования поставлены задачи в направлении развития адаптивного обучения и прокторинга. С помощью ИИ планируется автоматизировать работу по подбору учебного материала и способа преподавания, подходящих конкретному ученику, чтобы облегчить процесс усвоения материала всем категориям учащихся.

Функции прокторинга заключаются в контроле за учащимися во время сдачи экзамена. Роботу ставится задача следить за происходящим, фиксируя самые разные факторы, которые не <<видит>> глаз веб - камеры. Хотя, надо признать, в сфере образования не все можно доверить технике. Большую роль в обучении играет личность преподавателя, его харизма, умение выстроить правильные отношения с учениками.

Но даже среднестатистического учителя машина вряд ли заменит. По мнению Розы Лукин, профессора University College London, необходимо искать срединное решение. Задача не заключается в замене учителя компьютерной программой, а в улучшении процесса обучения. А это по плечу только человеку.

\subsection{Применение искусственного интеллекта в промышленной сфере}
    
В этой сфере применения систем искусственного интеллекта востребована возможность автоматизировать рабочие процессы. Чаще сегодня автоматизируют операции, выполняемые на конвейере.

Собственники крупных промышленных предприятий Японии, КНР, Соединенных Штатов, Германии, Швейцарии вкладывают огромные суммы в переоборудование заводов и фабрик. 

\subsection{Искусственный интеллект в сельском хозяйстве}
    
Сельское хозяйство также относится к основным сферам применения искусственного интеллекта. Предприятия используют ИИ для обнаружения и удаления на полях сорняков, выявления заболеваний культур, распознавания вредных насекомых, экономного распределения на площадях пестицидов и удобрений в необходимых количествах. Кроме того, системы отслеживают изменения параметров окружающей среды – температуры воздуха, влажности воздуха и т.д.

В ближайшем будущем планируется оснастить сельское хозяйство следующими устройствами:
    
\begin{itemize}
    \item{Беспилотными летательными аппаратами или дронами, которые с помощью GPSи радаров проводят обработку посевов химикатами и занимаются аэрофотосъемкой.}
    \item{Роботами для сбора урожая.}
\end{itemize}

В отличие от зерноуборочных комбайнов роботизированное приспособление для сбора клубники создано не Машинами с ИИ для борьбы с сорняками. Задачей устройства Hortibot, созданного Орхусским университетом (AarhusUniversitet) в Дании, является распознавание и устранение сорняков с помощью механических приспособлений и точечной обработки гербицидным составом. Этому роботу прочат большое будущее, ведь его применение значительно экономит средства и облегчает труд работников. Вскоре мы также увидим роботов, специализирующихся на вредителях и болезнях растений.

Аналитики Energias Market Research прогнозируют рост рынка ИИ в сельском хозяйстве на 24,3 процента. Наиболее активно будут применять автоматизацию в Соединенных Штатах и странах Азиатско-Тихоокеанского региона. Основными игроками здесь являются Agworld, Farmlogs, Cropx, Microsoft, AGCO и некоторые другие.

\subsection{Применение ИИ в сфере дорожного движения}
    
Оснащение логистической сферы устройствами с ИИ значительно снизит затраты времени на обработку гигантского объема данных. Система может объединить все внешние устройства, например, светофоры и отслеживать погодные условия, плотность автомобильного потока, количество и местоположение ДТП. На основе анализа данных о текущей обстановке ИИ сможет регулировать движение в городе, чтобы водители вовремя объезжали пробки, места ремонта и т.п.

Автоматизация проникает в такую сферу, как вождение автомобилей. Уже сегодня <<Тесла>> продает подписку на автопилот по всей США. Нас неизбежно ждет массовый переход на машины автономного типа, которым не нужен водитель.

\subsection{Применение ИИ в бытовой сфере}
    
Умные дома, разработанные Amazon, Apple, Google и Яндекс, предназначены для использования в разных странах мира. Раньше они имели ограниченный функционал, который представлял собой игру в <<города>>, будильник и управление домашними устройствами. Однако, с подключением GPT, умные домашние системы значительно усовершенствовались и теперь имеют возможности, которые ранее были не доступны.

Сейчас пользователи только начинают разбираться с функционалом, но потенциал просто огромен.

В скором времени с домом можно будет взаимодействовать, как с полноценным членом семьи. Он сможет приготовить утром костюм, разбудить, сделать заказ доставки продуктов, следить за микроклиматом в помещении, напоминать о времени стирки и уборки. Все это позволит меньше беспокоиться о бытовых вещах и рутинных обязанностях.

\subsection{Основные разработчики систем искусственного интеллекта}
   
Изучением возможностей искусственного интеллекта занимаются и лидеры информационного бизнеса наподобие Google и Amazon, и более мелкие участники рынка.

Больших успехов в плане разработки ИИ достигают даже неизвестные большинству компании, например, BotsCrew, которая разрабатывает чат-боты для Telegram и других сервисов. А такие фирмы, как InData Labs, nexocode создают программы для выполнения аналитических задач, обработки данных, статистических отчетов по запросам компаний-заказчиков.

\section{Недостатки и проблемы, связанные с использованием искусственного интеллекта}

\subsection{Доступность данных}
    
Часто, данные представлены в изоляции в компаниях или непоследовательно и низкое качество, которое представляет собой серьезную проблему для компаний, стремящихся создавать ценности из ИИ. Чтобы преодолеть этот барьер, жизненно важно с самого начала разработать четкую стратегию, чтобы иметь возможность извлекать данные организованным и последовательным образом.\cite{N14}

\subsection{Отсутствие квалифицированных специалистов}
    
Еще одно препятствие, которое часто возникает на уровне предприятия при внедрении ИИ, – это нехватка профилей с навыками и опытом в реализации этого типа. В этих случаях очень важно иметь профессионалов, которые уже работали над проектами такого же размера.
    
\subsection{Стоимость и сроки реализации ИИ-проектов}
    
Стоимость реализации, как с точки зрения сроков, так и с точки зрения экономики, является очень важным фактором при принятии решения о реализации этого типа проекта. Компании, которым не хватает внутренних навыков или незнакомы с системами ИИ, должны рассмотреть возможность передачи на аутсорсинг как внедрения, так и обслуживания, чтобы получить успешные результаты от своего проекта.

\subsection{Отсутсвие правового регулирования}

Нейросети, как и искусственный интеллект в целом, пока не имеют полноценного правового регулирования, что может привести к различным негативным последствиям. Проблема ответственности за действия систем ИИ ― самая обсуждаемая, когда речь идет о применении. 

Так, к примеру Илон Маск обратил внимание на опасности, заявив: <<Искусственный интеллект опаснее неправильного проектирования самолётов и производства плохих автомобилей. Потому что у него есть потенциал разрушения цивилизации>>

\section{Перспективы развития искусственного интеллекта}

\subsection{Глубокое обучение}
    
Эта техника, основанная на искусственных нейронных сетях, позволяет ИИ эффективно обрабатывать и анализировать большие объемы данных. Благодаря глубокому обучению, ИИ может улучшить свои функции распознавания образов и речи, а также улучшить свою способность к анализу и принятию решений на основе данных.\cite{N5}
    
Глубокое обучение уже нашло применение в различных областях жизни, включая медицину, финансы, розничную торговлю, автомобильную промышленность и многие другие сферы. 

Например, в медицине эта технология используется для анализа медицинских изображений и диагностики заболеваний. В финансовой сфере она помогает улучшить анализ рисков и прогнозирование рынка. В розничной торговле глубокое обучение используется для персонализации рекомендаций и повышения уровня обслуживания клиентов. В автомобильной промышленности оно позволяет улучшить системы безопасности и создать более удобные автомобили
    
\subsection{Разработка автономных систем}

Автономные системы, такие как беспилотные летательные аппараты, автономные автомобили и роботы, уже используются в некоторых областях, но дальнейшее развитие этой области позволит создавать более сложные и полезные автономные системы. Это возможно благодаря комбинации глубокого обучения с другими технологиями ИИ, такими как машинное зрение и обработка естественного языка.

Разработка автономных систем может иметь огромное значение для различных сфер деятельности. Например, в сфере медицины автономные роботы могут проводить сложные операции, не требуя наличия врачей. В промышленности они могут исполнять опасные работы, не подвергая человека риску. В транспортной сфере автономные автомобили могут улучшить безопасность на дорогах и сократить число ДТП.
    
\subsection{Создание сетей ИИ}
    
Еще одной перспективной областью развития ИИ является создание сетей ИИ, которые смогут самостоятельно обучаться и совершенствоваться. Это позволит ИИ стать еще более автономным и эффективным, поскольку он сможет самостоятельно адаптироваться к новым ситуациям и задачам.

Создание сетей ИИ, способных самостоятельно обучаться и совершенствоваться, уже является реальностью. Такие системы уже используются в таких областях, как игровая индустрия, где ИИ может самостоятельно учиться и улучшать свои игровые навыки. Однако, эту технологию можно применять и в других областях, например, в медицине, где ИИ может самостоятельно обучаться на основе анализа больших объемов медицинских данных, и в промышленности, где ИИ может самостоятельно улучшать производственные процессы.

\subsection{Разработка систем}

Еще одна возможность применения ИИ является разработка систем, которые могут помочь бороться с глобальными проблемами, такими как изменение климата, энергетическая эффективность и сокращение отходов. Например, с помощью ИИ можно создать системы управления потреблением энергии, которые могут значительно снизить энергетические затраты и уменьшить воздействие на окружающую среду. Также ИИ может использоваться для управления производственными процессами таким образом, чтобы сократить отходы и повысить эффективность использования ресурсов.

Кроме того, ИИ может помочь бороться с бедностью и голодом в мире. Например, благодаря анализу множества данных, связанных с производством и распределением продовольствия, ИИ может помочь оптимизировать эти процессы и повысить эффективность использования ресурсов. Также ИИ может использоваться для создания систем прогнозирования погоды, которые могут помочь сельскому хозяйству улучшить урожай и сократить риски связанные с погодными условиями.

\conclusion

В заключение можно отметить, что искусственный интеллект – это один из наиболее перспективных направлений развития науки и технологий. Это сфера, которая будет продолжать быстро развиваться и вносить много изменений в нашу жизнь. Сегодня искусственный интеллект присутствует в самых разных сферах, от бизнеса до медицины, и его значение и важность будут только расти. При этом, не следует забывать, что искусственный интеллект – это инструмент, который должен служить человеку, а не замещать его. Поэтому важно развивать и использовать интеллектуальные технологии в соответствии с высокими этическими нормами и принципами, чтобы они действительно помогали нам улучшать нашу жизнь и решать актуальные проблемы.

\bibliographystyle{gost780uv}
\bibliography{thesis}
\end{document}
