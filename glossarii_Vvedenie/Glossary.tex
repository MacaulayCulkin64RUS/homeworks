\documentclass{article}
\usepackage[russian, english]{babel}
\usepackage[utf8]{inputenc}
\usepackage[T2A]{fontenc}

\usepackage{graphicx} % Required for inserting images

\title{gloss}
\author{Константин Иллипуров}
\date{May 2023}

\begin{document}

%\maketitle

\section{Русский язык}
\begin{enumerate}
    \item Агрегация --- это процесс сбора данных или функций из разных источников, чтобы объединить их в один компактный элемент.
    \item Аджайл (Agile) --- это итеративный подход к управлению проектами и разработке программного обеспечения, который помогает командам быстрее и с меньшими проблемами поставлять ценность клиентам. (Барабанов Никита)
    \item Аппроксимация или приближение --- научный метод, состоящий в замене одних объектов другими, в каком-то смысле близкими к исходным, но более простыми. (Алексей Кузьмин) 
    \item Архитектурный стиль микросервисов --- это подход, при котором система строится как набор независимых и слабосвязанных сервисов, которые можно создавать используя различные языки программирования и технологии хранения данных. (Барабанов Никита)
    \item Баг (bug) --- ошибка в программе или в системе, из-за которой программа выдает неожиданное поведение и, как следствие, результат. (Барабанов Никита)
    \item Брокер сообщений (message broker) --- архитектурный паттерн в распределённых системах; приложение, которое преобразует сообщение по одному протоколу от приложения-источника в сообщение протокола приложения-приёмника, тем самым выступая между ними посредником. (Барабанов Никита)
    \item Бэкграунд опыт --- проффесиональный опыт, связанный (в данном случае) с it сферой (Алексей Кузьмин)
    \item Валидация --- процесс оценки того, насколько система (программа, устройство) соответствует требованиям по ее назначению.
    \item Веб-фреймворк --- это программный инструмент или набор инструментов, который предоставляет разработчикам веб-приложений стандартную структуру и набор абстракций для создания веб-приложений.
    \item Верификация --- это процесс оценки того, насколько система (программа, устройство) по итогам некоторого этапа ее разработки соответствует условиям, заданным в начале этапа. (Максим из Бюро)
    \item Гибридный формат работы --- это вариант, когда можно совмещать онлайн и офлайн работу.
    \item Деплой (deploy) --- это развертывание и запуск веб-приложения или сайта в его рабочей среде, то есть на сервере или хостинге. (Барабанов Никита)
    \item Директория  --- структура каталогов (папок), которые используются для организации и хранения файлов на компьютере или в сети.
    \item Драйвер  --- это программа, которая позволяет компьютеру взаимодействовать с определенным аппаратным устройством, таким как принтер, сканер, сетевая карта и т.д.
    \item Инкапсуляция --- это концепция программирования, означающая, что данные и функции, которые с ними связаны, скрыты от внешнего мира и доступны только через специальные методы класса.
    \item ИНС --- это  математическая модель, а также её программное или аппаратное воплощение, построенная по принципу организации и функционирования биологических нейронных сетей. (Максим из Бюро)
    \item Кластеризация (cluster analysis) --- задача группировки множества объектов на подмножества (кластеры) таким образом, чтобы объекты из одного кластера были более похожи друг на друга, чем на объекты из других кластеров по какому-либо критерию. (Алексей Кузьмин)
    \item Класс --- в объектно-ориентированном программировании, модель для создания объектов определённого типа, описывающая их структуру и определяющая алгоритмы для работы с этими объектами.
    \item Компилятор --- это программа, которая преобразует исходный код программы, написанного на языке программирования, в машинный код, который может быть понят компьютером.
    \item Контейнеризация --- это технология, которая помогает запускать приложения изолированно от основной операционной системы. (Барабанов Никита)
    \item Куки (cookies) --- это небольшие текстовые файлы, которые создаются веб-сервером и отправляются на компьютер пользователя в целях хранения определенной информации о посещенном сайте или о действиях пользователя на этом сайте.
    \item Лог --- текстовый файл с информацией о действиях программного обеспечения или пользователей, который хранится на компьютере или сервере. (Барабанов Никита)
    \item Логирование --- запись логов (Барабанов Никита)
    \item Локальный сервер --- это сервер, установленный и работающий только на одном компьютере или внутренней сети, который может обеспечивать доступ к веб-страницам, базам данных, приложениям и другим ресурсам и сервисам в пределах этой сети или компьютера.
    \item Массив --- это структура данных, которая представляет собой набор элементов одного типа, расположенных в памяти компьютера последовательно и доступных по индексу.
    \item Метод чёрного ящика --- метод исследования таких систем, когда вместо свойств и взаимосвязей составных частей системы, изучается реакция системы, как целого, на изменяющиеся условия. (Максим из Бюро)
    \item Микропроцессоры --- процессор, реализованный в виде одной микросхемы или комплекта из нескольких специализированных микросхем. (Батраева Инна Александровна)
    \item Микросервис --- изолированная часть приложения.
    \item Монолитное приложение --- единый централизованный модуль (Позволяет проводить быстрее сквозное тестирование, имеет более удобную отладку и другие плюсы).
    \item Метод чёрного ящика --- метод исследования таких систем, когда вместо свойств и взаимосвязей составных частей системы, изучается реакция системы, как целого, на изменяющиеся условия. (Максим из Бюро)
    \item Метаданные --- это данные, которые описывают другие данные и позволяют классифицировать, хранить, обрабатывать и передавать информацию.
    \item Нейросети --- математическая модель, а также её программное или аппаратное воплощение, построенное по принципу организации и функционирования биологических нейронных сетей. (Алексей Кузьмин)
    \item Облако (cloud)  --- это модель предоставления информационных технологий, при которой ресурсы (включая вычислительные мощности, сети, хранилища данных, приложения и сервисы) доступны через интернет в виде услуг.
    \item Парсинг --- это процесс преобразования структурированных данных из одного формата в другой, с целью автоматической обработки, анализа и использования этой информации.
    \item Плагин (plugin) --- это программное обеспечение, которое обычно используется в рамках другого программного продукта, добавляющее специальные функции или возможности этому продукту.
    \item Пуш (push) --- один из способов распространения информации в Интернете, когда данные поступают от поставщика к пользователю на основе установленных параметров.(Барабанов Никита)
    \item Реестр (Registry) --- это централизованная база данных в операционной системе Windows, где хранятся настройки, параметры и конфигурационные данные программ и компонентов, используемых в системе.
    \item Рекрут (recruiter) --- это специалист по подбору и найму сотрудников. (Ростислав Сибинтек; Алексей Кузьмин)
    \item Репозиторий (repository) --- место, где хранятся и поддерживаются какие-либо данные. Чаще всего данные в репозитории хранятся в виде файлов, доступных для дальнейшего распространения по сети.(Барабанов Никита)
    \item РНС (Recurrent neural network) --- вид нейронных сетей, где связи между элементами образуют направленную последовательность.
    \item САБ (SAP) --- это автоматизированная система, предлагающая комплекс решений для выстраивания общего информационного пространства на базе предприятия и эффективного планирования ресурсов и рабочих процессов. (Ростислав Сибинтек)
    \item Сверточные сети  (Convolutional Neural Networks) --- это тип нейронных сетей, которые применяются в обработке и анализе изображений и видео. (Алексей Кузьмин)
    \item СКР (Система Контроля Редактирования) --- это программное обеспечение, которое используется для управления изменениями в коде программы. СКР позволяет отслеживать изменения, сохранять и откатывать изменения. (Алексей Кузьмин)
    \item Трассировка --- процесс пошагового выполнения программы.(Максим из Бюро)
    \item Фреймворк (framework) --- программная платформа, определяющая структуру программной системы; программное обеспечение, облегчающее разработку и объединение разных компонентов большого программного проекта.(Алексей Кузьмин)
    \item Хост --- это компьютер или другое устройство, которое является сервером для других устройств или программ. Хост может предоставлять доступ к файлам, информации и ресурсам в сети, а также обеспечивать связь и передачу данных между устройствами.
    \item Эпоха обучения --- один проход по всем обучающим данным в алгоритме машинного обучения (Алексей Кузьмин)
\end{enumerate}

\section{Английский язык}
\begin{enumerate}
    \item ARP (Address Resolution Protocol) таблица --- это таблица в компьютерной сети, содержащая информацию о соответствии IP-адресов сетевых устройств и их MAC-адресов. 
    \item Business intelligence (BI) --- обозначение компьютерных методов и инструментов для организаций, обеспечивающих перевод транзакционной деловой информации в человекочитаемую форму, а также средства для массовой работы с такой обработанной информацией. (Алексей Кузьмин)
    \item ChatGPT --- чат-бот с искусственным интеллектом, разработанный компанией OpenAI и способный работать в диалоговом режиме, поддерживающий запросы на естественных языках.
    \item Deep learning  --- совокупность методов машинного обучения, основанных на обучении представлениями, а не специализированных алгоритмах под конкретные задачи. (Алексей Кузьмин)
    \item DevOps (development & operations) --- методология автоматизации технологических процессов сборки, настройки и развёртывания программного обеспечения. (Барабанов Никита)
    \item Data mining --- собирательное название, используемое для обозначения совокупности методов обнаружения в данных ранее неизвестных, нетривиальных, практически полезных и доступных интерпретации знаний, необходимых для принятия решений в различных сферах человеческой деятельности. (Алексей Кузьмин)
    \item ERP (enterprise resource planning) --- это планирование ресурсов предприятия. (Ростислав Сибинтек)
    \item GPT (Generative pre-trained transformer) --- это тип нейронных языковых моделей, впервые представленных компанией OpenAI, которые обучаются на больших наборах текстовых данных, чтобы генерировать текст, схожий с человеческим. (Алексей Кузьмин)
    \item GAN (Generative Adversarial Network) --- алгоритм машинного обучения без учителя, построенный на комбинации из двух нейронных сетей, одна из которых (сеть G) генерирует образцы, а другая (сеть D) старается отличить правильные («подлинные») образцы от неправильных (Алексей Кузьмин)
    \item Getter --- это метод класса, который используется для получения значения одного из свойств объекта или экземпляра класса.
    \item MSA (Measurement System Analysis) --- это метод, призванный дать заключение относительно приемлемости используемой измерительной системы через количественное выражение её характеристик. (Барабанов Никита)
    \item MLOps --- это набор практик нацеленных на надежное и эффективное развертывание и поддержание моделей машинного обучения на производстве.(Барабанов Никита)
    \item SOA (service-oriented architecture) --- модульный подход к разработке программного обеспечения, базирующийся на обеспечении удаленного по стандартизированным протоколам использования распределённых, слабо связанных, легко заменяемых компонентов (сервисов) со стандартизированными интерфейсами.(Барабанов Никита)
    \item UI/UX UX --- это User Experience. То есть это то, какой опыт/впечатление получает пользователь от работы с вашим интерфейсом; UI --- это User Interface --- то, как выглядит интерфейс и то, какие физические характеристики приобретает. (Барабанов Никита)
    \item UML (Unified Modeling Language)  --- язык графического описания для объектного моделирования в области разработки программного обеспечения, для моделирования бизнес-процессов, системного проектирования и отображения организационных структур.
\end{enumerate}

\end{document}

