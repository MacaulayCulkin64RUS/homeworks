\documentclass[bachelor,och,referat]{SCWorks_corrected}
\usepackage{graphicx} 

\usepackage[english,russian]{babel}

\usepackage[sort,compress]{cite}
\usepackage{amsmath}
\usepackage{amssymb}
\usepackage{amsthm}
\usepackage{fancyvrb}
\usepackage{longtable}
\usepackage{array}
\usepackage[english,russian]{babel}
\usepackage{tempora}
\usepackage{url}
\usepackage{hyperref}


\newcommand{\eqdef}{\stackrel {\rm def}{=}}

\newtheorem{lem}{Лемма}



\begin{document}
% Кафедра (в родительном падеже)
\chair{информатики и программирования}

% Тема работы
\title{Физика низких температур}

% Курс
\course{1}

% Группа
\group{111}

% Факультет (в родительном падеже) (по умолчанию "факультета КНиИТ")
\department{факультета КНиИТ}

% Специальность/направление код - наименование
\napravlenie{02.03.02 "--- Фундаментальная информатика и информационные технологии}
%\napravlenie{02.03.01 "--- Математическое обеспечение и администрирование информационных систем}
%\napravlenie{09.03.01 "--- Информатика и вычислительная техника}
%\napravlenie{09.03.04 "--- Программная инженерия}
%\napravlenie{10.05.01 "--- Компьютерная безопасность}

% Для студентки. Для работы студента следующая команда не нужна.
%\studenttitle{Студентки}

% Фамилия, имя, отчество в родительном падеже
\author{Иллипурова Константина Дмитриевича}

% Год выполнения отчета
\date{2023}

\maketitle
\tableofcontents

\intro
Физика низких температур "--- раздел физики, занимающийся изучением физических свойств твердых тел и явлений, протекающей при очень низких температурах. 

Физика низких температур используется во многих научных областях, включая:\begin{enumerate}
    \item Физика конденсированного состояния материи
    
    изучение свойств материалов при низких температурах, которые могут быть использованы для создания новых материалов с улучшенными свойствами.

    \item Астрофизика
    
    исследование свойств космических объектов, включая звезды, планеты и галактики, при помощи космических телескопов и наземных обсерваторий.

    \item Квантовая электродинамика
    
    изучение взаимодействия частиц при экстремальных условиях, включая низкие температуры, которые помогают определить точность измерения фундаментальных констант и поиск новой физики.

    \item Квантовые вычисления
    
    использование свойств квантовых систем для создания быстрых и эффективных вычислительных систем с низкими температурами.

    \item Медицина
    
    использование низких температур для консервации тканей и органов для трансплантации, исследований белков и лекарственных препаратов.

    \item Технология
    
    применение низкотемпературных технологий для производства полупроводниковых изделий и электронных устройств с высокой производительностью и энергоэффективностью.

\end{enumerate}

\section{Экскурс в историю}

Первым систематически исследовать низкотемпературные проблемы и возможности ожижения газов начал М.Фарадей. Он показал, что многие газы, например хлор, диоксид серы и аммиак, могут быть ожижены и при этом достигаются низкие температуры (до -110° С). Но многие другие газы не поддавались ожижению его методами даже при крайне высоких давлениях, за что позднее получили название постоянных газов.

В 1887 К.Ольшевскому и З.Врублевскому и Дж.Дьюару удалось получить в жидком виде многие постоянные газы в таких количествах, которые позволяли провести точные измерения и установить их низкотемпературные свойства.

В 1894 Г.Камерлинг "= Оннес построил установку для ожижения воздуха, а в 1895 У. Гемпсон и К. фон Линде независимо друг от друга разработали новый метод ожижения воздуха, а затем более совершенные методы были найдены Ж.Клодом во Франции и К.Гейландтом в Германии. Этими работами был заложен фундамент промышленности разделения газов, в которой результаты низкотемпературных исследований нашли самое важное и самое широкое техническое применение.

Впервые ожижить водород удалось в 1888 Дж.Дьюару тем же методом, которым ранее Гемпсон ожижал воздух. Таким образом, к концу 19 в. были ожижены все постоянные газы, кроме гелия, и завершены измерения их параметров

Ожижение гелия с массой 4 осуществил Камерлинг "= Оннес методом, почти совпадающим с методом ожижения воздуха Линде. Этим было не только установлено существование жидкой фазы для всех газов, но и открыта новая важная область низких температур.

Позднее в 20-е годы многие ученые изучали свойства жидкого гелия, а в 1933 году Вильгельм Мейсснер и Роберт Охм придумали и создали первый сверхпроводник.(6)


%\begin{enumerate}
%\begin{itemize}
\section{Свойства веществ при низких температурах}

При низких температурах, когда интенсивность тепловых движений оказывается ослабленной, наблюдаются существенные изменения свойств вещества, такие как:\begin{enumerate}
    \item Сверхпроводимость "--- это свойство некоторых веществ, проявляющееся в том, что они при достижении определенной низкой температуры теряют электрическое сопротивление. Это позволяет передавать электрический ток без потерь энергии.

    \item Сверхжидкость "--- это свойство некоторых легких элементов, таких как гелий "= 4, проявляющееся в том, что они приближаются к абсолютному нулю, становятся жидкостью, которая может двигаться без трения о стенки емкости.

    \item Магнитное закрепление "--- это свойство некоторых веществ, проявляющееся в том, что их магнитные поля сохраняются на очень долгое время при низких температурах.

    \item Сверхтекучесть "--- это свойство некоторых жидких веществ, которые при крайне низких температурах могут двигаться без вязкости и трения о стенки емкости.

    \item Фазовый переход "--- это свойство, когда переход из одной фазы (например, из жидкой в газообразную или твердую) происходит при определенной температуре и/или давлении.

    \item Сверхрешетка "--- это свойство, проявляющееся при экстремально низких температурах, когда атомы или молекулы вещества начинают формировать определенный рисунок, который сильно отличается от обычной кристаллической решетки.

\end{enumerate}

\section{Эксперименты и технологии}
Так как температура рассматриваемых объектов должна быть близка к абсолютному нулю, необходимо упомянуть технологии, с помощью которых добиваются такого результата. Помимо этого, физика низких температур включает в себя множество научных экспериментов, направленных на изучение свойств и поведения веществ при очень низких температурах.
\subsection{Эксперименты}\begin{enumerate}
    \item Изучение сверхпроводимости и сверхтекучести "= это свойства некоторых веществ, проявляющиеся при крайне низких температурах. Эксперименты в этой области включают измерение электрических и термических свойств сверхпроводников и сверхтекучих жидкостей.
    \item Изучение явления Бозе "= Эйнштейна "--- это квантово-механический эффект, проявляющийся при очень низких температурах и в который вовлечены большое количество атомов и молекул. Эксперименты в этой области включают создание и изучение конденсатов Бозе-Эйнштейна.
    \item Изучение свойств жидкого гелия "--- это вещество, которое проявляет сверхтекучесть при крайне низких температурах. Эксперименты в этой области включают изучение свойств жидкого гелия при различных условиях, а также исследование его поведения в магнитном поле.
    \item Изучение свойств криогенных материалов "--- это материалы, которые могут сохранять свои свойства при очень низких температурах. Эксперименты в этой области включают измерение механических, термических и электрических свойств криогенных материалов в зависимости от температуры.
\end{enumerate}

\subsection{Технологии охлаждения веществ до низких температур}\begin{enumerate}
    \item Криогенные резервуары "--- это емкости, в которых хранятся жидкие гелий и азот при низких температурах. Они используются в лабораториях для проведения экспериментов с низкотемпературными материалами.
    \item Криостаты "--- это устройства, которые служат для создания и поддержания очень низких температур. Они могут быть использованы для изучения сверхпроводимости, магнитных свойств и других явлений, проявляющихся при очень низких температурах. 
    \item Криогенные насосы "--- это устройства, которые служат для откачки газов из криогенных систем и создания вакуума. Они наиболее часто используются, когда необходимо создать очень низкое давление в системах. 
    \item Криотермические платы "--- это устройства, использующие сверхпроводники и пассивные элементы охлаждения, которые создают равномерную низкотемпературную поверхность. Они могут быть использованы в экспериментальных установках для создания равномерной температуры, необходимой для проведения определенных экспериментов. 
    \item Сверхпроводящие магниты "--- это устройства, использующие магниты, изготовленные из сверхпроводников и охлаждаемые жидким гелием или азотом. 
\end{enumerate}

\section{Практические применения и перспективы}
С помощью физики низких температур, мы можем создавать искусственные кристаллы, для производства полупроводниковых приборов, генерируем сверхпроводящие материалы, которые потом могут использоваться для создания магнитов и ускорителей частиц. У людей появляется возможность создавать и исследовать новые материалы с экстремальными свойствами, к примеру свехпластик или сверхпроводящие материалы высокой температуры. Будущее "= квантовые компьютеры "= невозможны без физики низкой температуры. Этот раздел касается даже топливных элементов, так как позволяет при экстремальных температурах обеспечивать более безопасно и эффективно ядерные станции.

Перспективы развития физики низких температур связаны с развитием новых технологий и применений. Одной из основных областей развития является разработка новых сверхпроводящих материалов, которые могут быть использованы для создания более эффективных электротехнических устройств. В будущем сверхпроводимость станет обширно применяться в энергетике, индустрии, на транспорте и значительно обширнее в медицине и электронике. В электронике сверхпроводимость найдет применение в компьютерных разработках. 

Потенциально выгодное промышленное использование сверхпроводимости связано с генерированием и передачей электричества. Еще одно перспективное использование сверхпроводников "--- в генераторах тока и электродвигателях. С развитием СП "= технологий сверхпроводящие движки найдут применение и в самолетах и на автотранспорте.

\conclusion
Физика низких температур является является основой для многих современных технологий и приложений, таких как:\begin{itemize}
    \item сверхпроводимость
    \item криогенные насосы
    \item криокондиционирование
    \item криозаборы\nocite{*}
    \item криогенная медицина
\end{itemize} 
Техника низкотемпературного ожижения позволяет получать из воздуха чистый кислород и чистый азот. Чистый кислород применяется в медицине, авиации и ракетно-космической технике, для сварки и резки стали, в доменных печах и бессемеровских конвертерах (для повышения выхода стали). Инертные газы, такие, как неон и аргон, широко применяемые в электрических лампах всех видов и при электросварке, в чистом виде могут быть получены только низкотемпературными (криогенными) методами.
%\begin{thebibliography}{11}

%\bibitem{Ione} The origin of Spacewar J. M. Graetz: 1981 issue of Creative Computing magazine. \\ 
%URL:  \url{https://www.wheels.org/spacewar/creative/SpacewarOrigin.html} (Дата обращения: 06.05.2023) \\
%\bibitem{Itwo} Краткая история развития игровых движков. Хабр. \\ URL: 
%\url{https://habr.com/ru/companies/miip/articles/314502/} (Дата обращения: 06.05.2023)\\
%\bibitem{Ithree} Как создаются видеоигры: процесс разработки игры. IT and Digital.\\ URL
%\url{https://itanddigital.ru/videogame} (Дата обращения 07.05.2023) \\
%\bibitem{Ifour} Модели жизненного цикла разработки игр. 3dgame-creator. URL \\ 
%https://3dgame-creator.ru/catalog/uroki/dlya-novichkov/modeli-zhiznennogo-cikla-razrabotki-igr/ (Дата обращения 07.05.2023)\\
%\bibitem{Ifive} Main component of game development stages. Argentics.io. \\
%URL: \url{https://www.argentics.io/main-component-of-game-development-stages} (Дата обращения 07.05.2023). 
%\bibitem{Isix} Step"=by"=step process of game development pre-production phase. Room8 Studio. \\
%URL: \url{https://room8studio.com/news/game-pre-production-core-steps/} (Дата обращения 07.05.2023). \\
%\bibitem{Iseven} Прототипирование в геймдеве. Spiiin's blog. \\
%URL: \url{https://spiiin.github.io/blog/2537188794/} (Дата обращения 07.05.2023)\\
%\bibitem{Ieight} Rapid Game Prototyping: Why is it Important in Game Development? Starloop. \\
%URL: https://starloopstudios.com/rapid-game-prototyping-why-is-it-important-in-game-development/ (Дата обращения 07.05.2023)\\
%\bibitem{Inine} Video game development. Wikipedia. \\
%URL: \url{https://en.wikipedia.org/wiki/Video_game_development} (Дата обращения 09.05.2023)\\
%\bibitem{Iten} Code freeze. Tutorialspoint. \\
%URL: \url{https://www.tutorialspoint.com/software_testing_dictionary/code_freeze.htm} (Дата обращения 09.05.2023) \\
%\bibitem{Ieleven} Что это - бета-версия и зачем она нужна? Autogear \\
%URL: \url{https://autogear.ru/article/182/201/chto-takoe-beta-versiya-i-zachem-ona-nujna/} (Дата обращения 09.05.2023)\\
%\end{thebibliography}
\bibliographystyle{gost780uv}
\bibliography{thesis}
\end{document}
